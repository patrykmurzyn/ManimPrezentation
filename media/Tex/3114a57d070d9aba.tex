\documentclass[preview]{standalone}

\usepackage[english]{babel}
\usepackage{amsmath}
\usepackage{amssymb}

\begin{document}

\begin{center}
Z algorytmu Dijkstry można skorzystać przy obliczaniu najkrótszej drogi do danej miejscowości. Wystarczy przyjąć, że każdy z punktów skrzyżowań dróg to jeden z wierzchołków grafu, a odległości miedzy punktami to wagi krawędzi. Jest cz\k esto używany w sieciach komputerowych, np. przy trasowaniu
\end{center}

\end{document}
